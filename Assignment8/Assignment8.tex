\documentclass[journal,12pt,twocolumn]{IEEEtran}
\usepackage{amsthm}
\allowbreak
\usepackage{setspace}
\usepackage{gensymb}
\singlespacing
\usepackage[cmex10]{amsmath}
\usepackage{caption}
\usepackage{amsthm}



\DeclareUnicodeCharacter{2212}{-}
\usepackage{tikz}
\usepackage{pgfplots}

\usepackage{mathrsfs}
\usepackage{txfonts}
\usepackage{stfloats}
\usepackage{bm}
\usepackage{cite}
\usepackage{cases}
\usepackage{subfig}

\usepackage{longtable}
\usepackage{multirow}

\usepackage{enumitem}
\usepackage{mathtools}
\usepackage{steinmetz}
\usepackage{tikz}
\usepackage{circuitikz}
\usepackage{verbatim}
\usepackage{tfrupee}
\usepackage[breaklinks=true]{hyperref}
\usepackage{graphicx}
\usepackage{tkz-euclide}
\graphicspath{ {./images/} }
\usetikzlibrary{calc,math}
\usepackage{listings}
    \usepackage{color}                                            %%
    \usepackage{array}                                            %%
    \usepackage{longtable}                                        %%
    \usepackage{calc}                                             %%
    \usepackage{multirow}                                         %%
    \usepackage{hhline}                                           %%
    \usepackage{ifthen}                                           %%
    \usepackage{lscape}     
\usepackage{multicol}
\usepackage{chngcntr}

\DeclareMathOperator*{\Res}{Res}

\renewcommand\thesection{\arabic{section}}
\renewcommand\thesubsection{\thesection.\arabic{subsection}}
\renewcommand\thesubsubsection{\thesubsection.\arabic{subsubsection}}

\renewcommand\thesectiondis{\arabic{section}}
\renewcommand\thesubsectiondis{\thesectiondis.\arabic{subsection}}
\renewcommand\thesubsubsectiondis{\thesubsectiondis.\arabic{subsubsection}}


\hyphenation{op-tical net-works semi-conduc-tor}
\def\inputGnumericTable{}                                 %%

\lstset{
%language=C,
frame=single, 
breaklines=true,
columns=fullflexible
}
\begin{document}


\newtheorem{theorem}{Theorem}[section]
\newtheorem{problem}{Problem}
\newtheorem{proposition}{Proposition}[section]
\newtheorem{lemma}{Lemma}[section]
\newtheorem{corollary}[theorem]{Corollary}
\newtheorem{example}{Example}[section]
\newtheorem{definition}[problem]{Definition}

\newcommand{\BEQA}{\begin{eqnarray}}
\newcommand{\EEQA}{\end{eqnarray}}
\newcommand{\define}{\stackrel{\triangle}{=}}
\bibliographystyle{IEEEtran}
\raggedbottom
\setlength{\parindent}{0pt}
\providecommand{\mbf}{\mathbf}
\providecommand{\pr}[1]{\ensuremath{\Pr\left(#1\right)}}
\providecommand{\qfunc}[1]{\ensuremath{Q\left(#1\right)}}
\providecommand{\sbrak}[1]{\ensuremath{{}\left[#1\right]}}
\providecommand{\lsbrak}[1]{\ensuremath{{}\left[#1\right.}}
\providecommand{\rsbrak}[1]{\ensuremath{{}\left.#1\right]}}
\providecommand{\brak}[1]{\ensuremath{\left(#1\right)}}
\providecommand{\lbrak}[1]{\ensuremath{\left(#1\right.}}
\providecommand{\rbrak}[1]{\ensuremath{\left.#1\right)}}
\providecommand{\cbrak}[1]{\ensuremath{\left\{#1\right\}}}
\providecommand{\lcbrak}[1]{\ensuremath{\left\{#1\right.}}
\providecommand{\rcbrak}[1]{\ensuremath{\left.#1\right\}}}
\theoremstyle{remark}
\newtheorem{rem}{Remark}
\newcommand{\sgn}{\mathop{\mathrm{sgn}}}
\providecommand{\abs}[1]{$\left\vert#1\right\vert$}
\providecommand{\res}[1]{\Res\displaylimits_{#1}} 
\providecommand{\norm}[1]{$\left\lVert#1\right\rVert$}
%\providecommand{\norm}[1]{\lVert#1\rVert}
\providecommand{\mtx}[1]{\mathbf{#1}}
\providecommand{\mean}[1]{E$\left[ #1 \right]$}
\providecommand{\fourier}{\overset{\mathcal{F}}{ \rightleftharpoons}}
%\providecommand{\hilbert}{\overset{\mathcal{H}}{ \rightleftharpoons}}
\providecommand{\system}{\overset{\mathcal{H}}{ \longleftrightarrow}}
	%\newcommand{\solution}[2]{\textbf{Solution:}{#1}}
\newcommand{\solution}{\noindent \textbf{Solution: }}
\newcommand{\cosec}{\,\text{cosec}\,}
\providecommand{\dec}[2]{\ensuremath{\overset{#1}{\underset{#2}{\gtrless}}}}
\newcommand{\myvec}[1]{\ensuremath{\begin{pmatrix}#1\end{pmatrix}}}
\newcommand{\mydet}[1]{\ensuremath{\begin{vmatrix}#1\end{vmatrix}}}
\numberwithin{equation}{subsection}
\makeatletter
\@addtoreset{figure}{problem}
\makeatother
\let\StandardTheFigure\thefigure
\let\vec\mathbf
\renewcommand{\thefigure}{\theproblem}
\def\putbox#1#2#3{\makebox[0in][l]{\makebox[#1][l]{}\raisebox{\baselineskip}[0in][0in]{\raisebox{#2}[0in][0in]{#3}}}}
     \def\rightbox#1{\makebox[0in][r]{#1}}
     \def\centbox#1{\makebox[0in]{#1}}
     \def\topbox#1{\raisebox{-\baselineskip}[0in][0in]{#1}}
     \def\midbox#1{\raisebox{-0.5\baselineskip}[0in][0in]{#1}}
\vspace{3cm}
\title{AI1103: Assignment 8}
\author{Tanmay Garg \\CS20BTECH11063 EE20BTECH11048}
\maketitle
\newpage
\bigskip
\renewcommand{\thefigure}{\theenumi}
\renewcommand{\thetable}{\theenumi}
%
Download latex-tikz codes from 
%
\begin{lstlisting}
need
\end{lstlisting}
\section*{Problem CSIR UGC NET EXAM (Dec 2016), Q.109: }
$X_1,X_2,\ldots,X_n$ are independent and identically
distributed as $N(\mu, \sigma^2)$, $-\infty < \mu < \infty$, $\sigma^2 > 0$. Then
\begin{enumerate}
    \item $\sum_1^n\frac{(X_i-\bar{X})^2}{n-1}$ is the Minimum Variance Unbiased Estimate of $\sigma^2$\\
    \item $\sqrt{\sum_1^n\frac{(X_i-\bar{X})^2}{n-1}}$ is the Minimum Variance Unbiased Estimate of $\sigma$\\
    \item $\sum_1^n\frac{(X_i-\bar{X})^2}{n}$ is the Maximum Likelihood Estimate of $\sigma^2$\\
    \item $\sqrt{\sum_1^n\frac{(X_i-\bar{X})^2}{n}}$ is the Maximum Likelihood \qquad Estimate of $\sigma$
\end{enumerate}

\section*{Solution:}
The pdf for each random variable is same as they are all identical and independent Normal Distributions with same $\mu$ and $\sigma^2$.
\begin{align}
    f_X(x) = \frac{1}{\sqrt{2\pi\sigma^2}}\exp{\frac{(x-\mu)^2}{2\sigma^2}}
\end{align}
Let us take our maximum likelihood function for given random variable $X_i$
\begin{align}
    L(\mu ; \sigma | X_i) = \frac{1}{\sqrt{2\pi\sigma^2}}\exp{\frac{(X_i-\mu)^2}{2\sigma^2}}\label{gen_eq}
\end{align}
Since all the random variables are i.i.d
\begin{align}
    L(\mu ; \sigma | X_1,X_2,\ldots,X_n) = \prod_{i=1}^nL(\mu ; \sigma | X_i)\label{main_eq}
\end{align}
Let us denote:
\begin{align}
    L_m : L(\mu ; \sigma | X_1,X_2,\ldots,X_n)
\end{align}
Substituting \eqref{gen_eq} for each Random Variable in \eqref{main_eq}
\begin{align}
    L_m = \prod_{i=1}^n\frac{1}{\sqrt{2\pi\sigma^2}}\exp{\frac{(X_i-\mu)^2}{2\sigma^2}}
\end{align}
Taking natural log on both sides and simplifying
\begin{align}
    \ln{L_m} = \frac{-n}{2}\ln{2\pi} -n\ln{\sigma} - \sum_{i = 1}^n\frac{(X_i - \mu)^2}{2\sigma^2}
\end{align}
In order to find Maximum Likelihood we need to maximise $\mu$ and $\sigma$ w.r.t. all Random variables. Taking partial derivative w.r.t $\mu$ and taking $\sigma$ as constant 
\begin{align}
    \frac{\partial \ln{L_m}}{\partial \mu} = \sum_{i = 1}^n\frac{(X_i - \mu)}{\sigma^2}
\end{align}
The value for $\mu$ at which $L_m$ achieves maximum value is same in $\ln{L_m}$
\begin{align}
   \because \frac{\partial \ln{L_m}}{\partial \mu} &=0\\
   \therefore \sum_{i = 1}^n\frac{(X_i -\mu)}{\sigma^2}&=0
\end{align}
On simplifying the expression we get:
\begin{align}
    n\mu &= \sum_{i=1}^nX_i\\
    \mu &= \frac{1}{n}\sum_{i=1}^nX_i\label{mu_value}
\end{align}
Let us denote the value achieved in \eqref{mu_value} as $\bar{X}$. Taking partial derivative w.r.t $\sigma$ and taking $\mu$ as constant
\begin{align}
    \frac{\partial \ln{L_m}}{\partial \sigma} &= \frac{-n}{\sigma} + \sum_{i=1}^n\frac{(X_i - \mu)^2}{\sigma^3}
\end{align}
The value for $\sigma$ at which $L_m$ achieves maximum value is same in $\ln{L_m}$
\begin{align}
    \frac{\partial \ln{L_m}}{\partial \sigma} &= 0\\
    \frac{-n}{\sigma} + \sum_{i=1}^n\frac{(X_i - \mu)^2}{\sigma^3} &=0
\end{align}
Upon simplifying the expression
\begin{align}
\frac{n}{\sigma} &= \sum_{i=1}^n \frac{(X_i -\mu)^2}{\sigma^3}\\
{\sigma^2} &= \sum_{i=1}^n\frac{(X_i-\mu)^2}{n}\label{sig_value}
\end{align}
Substituting \eqref{mu_value} in \eqref{sig_value}
\begin{align}
    {\sigma^2} &= \sum_{i=1}^n\frac{(X_i-\bar{X})^2}{n}\\
    {\sigma} &= \sqrt{\sum_{i=1}^n\frac{(X_i-\bar{X})^2}{n}}
\end{align}
Hence \textbf{Option 3} and \textbf{Option 4} are correct
\end{document}